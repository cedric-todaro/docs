\documentclass[a4paper,10pt]{article}
\usepackage[T1]{fontenc}
% \usepackage[utf8]{inputenc}
\usepackage[french]{babel}
\usepackage[autolanguage]{numprint}
\usepackage{calc,blindtext,lastpage}
\usepackage{amsmath,amsfonts,amssymb,nccmath,unicode-math}

% --- Polices
\usepackage{fontspec,fontawesome5}
\setmainfont{Asap}
\usepackage{contour,ulem}
\renewcommand{\ULdepth}{1.5pt}
\renewcommand{\ULthickness}{0.5pt}
\contourlength{0.6pt}
\newcommand{\myuline}[1]{\uline{\phantom{#1}}\llap{\contour{white}{#1}}}
\setlength{\parindent}{0pt}
% \usepackage{mlmodern}
% \renewcommand{\familydefault}{\sfdefault}

% --- Geometry
\usepackage[left=1cm,right=1cm,top=1.35cm,bottom=1.75cm]{geometry}
\setlength{\headheight}{1.30cm}
\setlength{\headsep}{0.35cm}
\setlength{\footskip}{0.5cm}
\setlength{\fboxrule}{1pt}

% --- Graphics
\usepackage{xcolor,graphicx,tikz,pgf,pgfplots}
\pgfplotsset{compat=1.15}
\usetikzlibrary{arrows,backgrounds}
\usepackage{array,multirow,makecell,colortbl,multicol}
\setlength{\columnseprule}{1pt}
\def\columnseprulecolor{\color{black}}

% --- Headers et footers
\usepackage{fancyhdr}
\pagestyle{fancy}
\renewcommand{\headrulewidth}{1pt}
\fancyhead[L]{\small Nom - Prénom :\hfill\Classe}\fancyhead[C]{}\fancyhead[R]{}
\renewcommand{\footrulewidth}{1pt}
\fancyfoot[L]{\small\Titre}\fancyfoot[C]{}\fancyfoot[R]{\small\thepage/\pageref{LastPage}}

% --- macro titre / section
\newcommand\MonTitre[1]{\begin{center}\LARGE\bfseries\myuline{#1}\end{center}}
\renewcommand{\section}[1]{\large\textbf{\uline{#1}}\normalsize\par}

% --- macro math
\let\le\leqslant\let\ge\geqslant
\newcommand\R{\mathbb{R}}
\newcommand\D{\mathbb{D}}
\newcommand\Q{\mathbb{Q}}
\newcommand\Z{\mathbb{Z}}
\newcommand\N{\mathbb{N}}
\newcommand\C{\mathbb{C}}
\renewcommand\vec[1]{\overrightarrow{#1}}
\newcommand\coord[2]{\begin{pmatrix}#1\\#2\end{pmatrix}}
\newcommand\coordl[2]{\left(#1;#2\right)}
\newcommand\eqv{\Leftrightarrow}


% --- Titre et classe
\def\Titre{Suites numériques}
\def\Classe{1\up{ère} Spé}

% ---
\begin{document}
\MonTitre{Activité : \Titre}

\section{Activité 1 : des nombres et leurs places}
\begin{enumerate}
	\def\labelenumi{\arabic{enumi}.}
	\item {
	      Aurélia a acquis un portable. Comme elle a pris son abonnement au cours du mois, l'opérateur lui a offert deux heuresde communication au départ. Les mois suivants, elle a relevé sa consommation téléphonique, arrondie au dixième d'heures.
	      \begin{center}
		      \def\arraystretch{1.6}
		      \begin{tabular}{|c|c|c|c|c|c|c|c|c|c|c|c|c|c|c|c|c|c|}\hline
			      2,1h & 3,1h & 3,4h & 4,3h & 2h & 2,8h & 2h & 1,5h & 2,3h & 1,4h & 1h & 2h & 1,7h & 2,5h & 3h & 4h & 2h \\ \hline
		      \end{tabular}
	      \end{center}

	      \begin{flushright}
		      ...ce qui signifie qu'elle a consommée 2,1 h le premier mois, 3,1 h le deuxième, etc...
	      \end{flushright}

	      On note $u_n$ la consommation d'Aurélia le $n\up{ième}$ mois de son abonnement. Les 2 heures de consommation offertes au départ sont notées $u_0$. On a donc $u_0 = 2$.
	      \begin{enumerate}
		      \item Déterminer $u_1$, $u_2$, $u_3$ et $u_{15}$ .
		      \item Quelle est la consommation le neuvième mois ? le seizième mois ? Donner la notation correspondante.
	      \end{enumerate}
	      }\vspace*{2mm}

	\item {
	      On considère la suite des multiples de 7 : $0$, $7$, $14$, $21$, $\ldots$

	      On les note successivement $u_0$ , $u_1$ , $u_2$ , $u_3$ , $\ldots$ (ainsi $u_0 = 0$, $u_1 = 7$, etc...)
	      \begin{enumerate}
		      \item Donner les valeurs de $u_3$ , $u_4$ , $u_{25}$ , $u_{100}$ et $u_{286}$.
		      \item Exprimer $u_n$ en fonction de $n$.
	      \end{enumerate}
	      }\vspace*{2mm}

	\item {
	      À sa date anniversaire, Elora reçoit de sa grand-mère dix fois son âge en euros, auquel elle ajoute 25€.

	      On note $u_n$ le montant, en €, reçu par Elora pour son $n\up{ième}$ anniversaire.
	      \begin{enumerate}
		      \item Déterminer $u_1$ , $u_2$ et $u_5$ .
		      \item Déterminer la somme reçue par Elora pour ses $7$ ans et donner la notation correspondante.
		      \item Exprimer $u_n$ en fonction de $n$.
	      \end{enumerate}
	      }\vspace*{2mm}

	\item {
	      On a pu modéliser le nombre de ménages équipés d'un micro-ondes en fonction de l'année $n$, en prenant $n = 0$ en 1995.  On note $u_n$ le nombre de ménages en millions.

	      On a obtenu : $u_n =\dfrac{6n + 4}{n+2}$
	      \begin{enumerate}
		      \item Calculer le nombre de ménages équipés en 1995, en 2003 puis en 2011.
	      \end{enumerate}
	      }
\end{enumerate}
\vspace*{3mm}
\section{Activité 2 : des nombres obtenus par un procédé}
\begin{enumerate}
	\def\labelenumi{\arabic{enumi}.}
	\item {
	      On prend un nombre entier au départ et on imagine un jeu : on divise le nombre par $2$, on ajoute $1$ au résultat et on multiplie par $3$. On trouve un nouveau nombre et on recommence.

	      On choisit comme nombre initial $u_0 = 4$.
	      \begin{enumerate}
		      \item Calculer $u_1$ et $u_2$
		      \item Peut-on facilement calculer $u_{25}$ ? Pourquoi ?
		      \item Montrer que : $~u_{n+1}=\left(\dfrac{u_n}{2} + 1\right)\times 3$
	      \end{enumerate}
	      }\vspace*{2mm}

	\item {
	      Une ville a une population de $200\ 000$ habitants l'année $0$.

	      Chaque année, sa population augmente de $5\%$ par rapport à la population de l'année précédente, mais $15\ 000$ habitants déménagent et quittent la ville.

	      On note $~p_0 = 200\ 000~$ et $~p_n~$ la population de l'année $n$.
	      \begin{enumerate}
		      \item Calculer $p_1$ et $p_2$ .
		      \item Établir une formule donnant la population en fonction de la population de l'année précédente.
		      \item À l'aide d'une calculatrice, déterminer la population de la ville l'année $15$.
	      \end{enumerate}
	      }
\end{enumerate}

\end{document}
