\input{_ttcours.latex}

\def\Titre{Fonctions du 2\up{nd} degré}
\def\DevoirSur{}

\begin{document}
\DoTitle{}
\section{Définition}

\ul{Ex. :}

\begin{itemize}
    \item $f(x)=3x^{\textcolor{red}{2}}-7x+3$
    \item $g(x)=\dfrac{1}{2}x^{\textcolor{red}{2}}-5x+\frac{3}{5}$
    \item $h(x)=4-2x^{\textcolor{red}{2}}$
    \item $k(x)=(x-4)(5-2x)\quad=(-2x^{\textcolor{red}{2}}+13x-20)$
\end{itemize}
\par~\par
\ul{Contre-exemples :}

\begin{itemize}
    \item $m(x)=5x-3$ ...est une fonction polynôme de degré $1$ (fonction affine).
    \item $n(x)=5x^{\textcolor{blue}{4}}-7x^{3}+3x-8$ ...est une fonction polynôme de degré $4$.
\end{itemize}
\par~\par
\ul{\textbf{Déf. :}}

On appelle \textbf{fonction polynôme de degré 2} toute fonction $f$ définie sur $\mathbb{R}$ par une expression de la forme :

$$\boxed{f(x)=ax^{\textcolor{red}{2}}+bx+c}$$

Les coefficients $a$, $b$ et $c$ sont des réels donnés avec $a \neq 0$.
\par~\par
\ul{\textbf{Def. :}}

Les fonctions polynômes de degré 2 \textbf{étudiées dans ce chapitre} sont définies sur $\mathbb{R}$ par :

$$\boxed{x \mapsto ax^{2}}\quad\text{ou}\quad\boxed{x\mapsto ax^{2}+b}$$

\par$~$\par


\ul{Rem. :}

Une fonction polynôme du 2\up{nd} degré s'appelle également \textbf{trinôme}.

\section{Résoudre $f(x)=0$}

\ul{\textbf{Méthode : résoudre $ax^2+bx+c=0$}}

\begin{enumerate}
    \item Identifier $a$ , $b$ et $c$
    \item Calculer le discriminant : $\Delta=b^2-4ac$
    \item Calculer les solutions en fonction du signe de $\Delta$ :
          \begin{itemize}
              \item Si $\Delta<0$ --> Pas de solution dans $\R$
              \item Si $\Delta=0$ --> 1 solutions dans $\R$ : $x_0=\dfrac{-b}{2a}$
              \item Si $\Delta>0$ --> 2 solutions dans $\R$ : $x_1=\dfrac{-b-\sqrt{\Delta}}{2a}$ et $x_2=\dfrac{-b+\sqrt{\Delta}}{2a}$
          \end{itemize}
\end{enumerate}


\end{document}
