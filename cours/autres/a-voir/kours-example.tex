\documentclass[nothmnum,use boldface,runin]{lebhart}
\usepackage{ProjLib}
\UseLanguage{French}
\title{Banane}
\author{Prout}
\date{\today}

\geometry{a4paper,margin=1.5cm}
\usepackage{tikz}
\usepackage{graphicx}
\usepackage{lipsum}
\usepackage{tikzducks}
\usepackage{listings}

\lstset{%
  language=Python,
  basicstyle   = \ttfamily,
  keywordstyle =    \color{magenta},
  keywordstyle = [2]\color{orange},
  commentstyle =    \color{gray}\itshape,
  stringstyle  =    \color{cyan},
  numbers      = left,
  frame        = single,
  framesep     = 2pt,
  aboveskip    = 1ex
}

\begin{document}
\maketitle

\section{Prout}
\noindent\begin{minipage}{12cm}
    \begin{definition}\lipsum[1]\end{definition}
\end{minipage}\hfill
\begin{minipage}{5cm}
    \centering
    \begin{tikzpicture}
        \duck[think={Blub},bubblecolour=white!65!black]
    \end{tikzpicture}
\end{minipage}

\section{Prout}
\begin{assumption}\lipsum[1]\end{assumption}
\begin{axiom}\lipsum[1]\end{axiom}
\begin{conjecture}\lipsum[1]\end{conjecture}
\begin{convention}\lipsum[1]\end{convention}
\begin{corollary}\lipsum[1]\end{corollary}
\begin{definition}\lipsum[1]\end{definition}
\begin{definition-proposition}\lipsum[1]\end{definition-proposition}
\begin{definition-theorem}\lipsum[1]\end{definition-theorem}
\begin{example}\lipsum[1]\end{example}
\begin{exercise}\lipsum[1]\end{exercise}
\begin{fact}\lipsum[1]\end{fact}
\begin{hypothesis}\lipsum[1]\end{hypothesis}
\begin{lemma}\lipsum[1]\end{lemma}
\begin{notation}\lipsum[1]\end{notation}
\begin{observation}\lipsum[1]\end{observation}
\begin{problem}\lipsum[1]\end{problem}
\begin{property}\lipsum[1]\end{property}
\begin{proposition}\lipsum[1]\end{proposition}
\begin{question}\lipsum[1]\end{question}
\begin{remark}\lipsum[1]\end{remark}
\begin{theorem}\lipsum[1]\end{theorem}

\newpage
\section{Patate}
\lipsum[1]
\subsection{super Patate}
\lipsum[1]
\begin{itemize}
    \item plopdfsa
    \item plopdfsa 1
          \begin{enumerate}
              \item plopdfsa 2
              \item \begin{enumerate}
                        \item plosa 2
                        \item \begin{itemize}
                                  \item plosa 2
                                  \item plsdaafdsosa 2
                              \end{itemize}
                        \item plosa 2
                    \end{enumerate}
              \item plofsa 2
          \end{enumerate}
    \item plopdfsa 3
\end{itemize}
\lipsum[1]

\subsubsection{super Patate}
\lipsum[1]

\DNF

\DNF{vite vite vite}

\begin{lstlisting}
    from random import randint

    for i range(12):
        print(i)
\end{lstlisting}

\end{document}