%! TeX program = pdflatex
\def\Classe{2\nde}
\def\Titre{Fonctions et équations du 2nd degré}
\input{_tt.latex}
\begin{document}
\DoTitle
\printtoc
\section{2nd degré}
\subsection{Équation du 2nd degré}
\definition Soit l'équation $$\boxed{ax^2+bx+c=0}$$ avec $a\neq 0$. Cette équation est une équation du 2\nd degré.
\hligne
\exemple
\begin{itemize}
  \item $3x^2+2x-3=0$\qquad avec $a=3,~b=2$ et $c=-3$
  \item $x^2-3=0$\qquad avec $a=1,~b=0$ et $c=-3$
  \item $6-9x^2=0$\qquad avec $a=-9,~b=0$ et $c=6$
  \item $x(x-4)=0\eqv x^2-4x=0$\qquad avec $a=1,~b=-4$ et $c=0$
\end{itemize}

~

\remarque Si $a=0$ alors cette équation n'est plus une équation du 2\nd degré mais du 1\er degré.
\hligne
\methode Pour résoudre une équation du 2\nd degré, il faut :
\begin{enumerate}
  \item Identifier $a$, $b$ et $c$
  \item Calculer $\Delta=b^2-4ac$
  \item En fonction du signe de $\Delta$ :
    \begin{enumerate}
      \item Si $\Delta>0$ alors il existe $2$ solutions : $\boxed{x_1=\dfrac{-b-\sqrt{\Delta}}{2a}}$ et $\boxed{x_2=\dfrac{-b+\sqrt{\Delta}}{2a}}$
      \item Si $\Delta=0$ alors il existe $1$ solution : $\boxed{x_0=\dfrac{-b}{2a}}$
      \item Si $\Delta<0$ alors il n'existe pas de solution.
    \end{enumerate}
\end{enumerate}

\hligne

etc etc etc ... \vec{u}

%%%%%%%%%%%%%%%%%%%%%%%%%
\end{document}
