\input{_bac.latex}

\rhead{\small{Ploum}}
\lhead{\small{Ploum}}
\lfoot{\small{Polynésie}}
\rfoot{\small{\thepage~/~\pageref*{LastPage}}}
\cfoot{\small{4 mai 2022}}


\begin{document}
\quad
\begin{center}
	\huge\textbf{\decofourleft~Baccalauréat spécialité 2022~\decofourright}
\end{center}

\bigskip

\textbf{\textsc{Exercice 2}  \quad  7 points\hfill Thèmes : probabilités}

On considère une fonction $f$ définie et dérivable sur $[-2~;~2]$. Le tableau de variations de la fonction $f'$ dérivée de la fonction $f$ sur l'intervalle $[2~;~2]$ est donné par :

\begin{center}
	{\renewcommand{\arraystretch}{1.2}
		\psset{nodesep=3pt,arrowsize=2pt 3}  % paramètres
		\def\esp{\hspace*{1.5cm}}% pour modifier la largeur du tableau
		\def\hauteur{0pt}% mettre au moins 20pt pour augmenter la hauteur
		$\begin{array}{|c| *4{c} c|}
				\hline
				x                       & -2              & \esp & 0               & \esp & 2                    \\
				% \hline
				%f'(x) &  &  \pmb{-} & \vline\hspace{-2.7pt}0 & \pmb{+} & \\  
				\hline
				                        & \Rnode{max1}{1} &      &                 &      & \Rnode{max2}{-1}     \\
				\text{variations de }f' &                 &      &                 &      & \rule{0pt}{\hauteur} \\
				                        &                 &      & \Rnode{min}{-2} &      & \rule{0pt}{\hauteur}
				\ncline{->}{max1}{min} \ncline{->}{min}{max2}
				\rput*(-3.7,0.5){\Rnode{zero}{0}}
				\rput(-3.7,1.65){\Rnode{alpha}{-1}}
				% \ncline[linestyle=dotted, linecolor=blue]{alpha}{zero}
				% \rput*(-1.3,0.65){\Rnode{zero2}{\red 0}}
				% \rput(-1.3,1.7){\Rnode{beta}{\red \beta}}
				% \ncline[linestyle=dotted, linecolor=red]{beta}{zero2}
				\\
				\hline
			\end{array}$
	}
\end{center}


\textbf{\textsc{Exercice 2} \quad 7 points\hfill Thèmes : probabilités}

\medskip

Les douanes s'intéressent aux importations de casques audio portant le logo d'une certaine marque. Les saisies des douanes permettent d'estimer que:

\setlength\parindent{10mm}
\begin{itemize}
	\item[$\bullet~~$] 20\,\% des casques audio portant le logo de cette marque sont des contrefaçons ;
	\item[$\bullet~~$] 2\,\% des casques non contrefaits présentent un défaut de conception ;\item[$\bullet~~$] 10\,\% des casques contrefaits présentent un défaut de conception.
\end{itemize}
\setlength\parindent{0mm}

L'agence des fraudes commande au hasard sur un site internet un casque affichant le logo de la marque. On considère les évènements suivants:

\setlength\parindent{10mm}
\begin{itemize}
	\item[$\bullet~~$] $C$: \og le casque est contrefait \fg{} ;
	\item[$\bullet~~$] $D$: \og le casque présente un défaut de conception \fg{} ;
	\item[$\bullet~~$] $\overline{C}$ et $\overline{D}$ désignent respectivement les évènements contraires de $C$ et $D$.
\end{itemize}
\setlength\parindent{0mm}

Dans l'ensemble de l'exercice, les probabilités seront arrondies à $10^{-3}$ si nécessaire.

\bigskip

\textbf{Partie 1}

\medskip

\begin{enumerate}
	\item Calculer $P(C \cap D)$. On pourra s'appuyer sur un arbre pondéré.\index{arbre pondéré}
	\item Démontrer que $P(D) = 0,036$.
	\item Le casque a un défaut. Quelle est la probabilité qu'il soit contrefait ?
\end{enumerate}

\bigskip

\textbf{Partie 2}

\medskip

On commande $n$ casques portant le logo de cette marque. On assimile cette expérience à
un tirage aléatoire avec remise. On note $X$ la variable aléatoire qui donne le nombre de casques présentant un défaut de conception dans ce lot.\index{loi binomiale}

\medskip

\begin{enumerate}
	\item Dans cette question, $n = 35$.
	      \begin{enumerate}
		      \item Justifier que $X$ suit une loi binomiale $\mathcal{B}(n,~p)$ où $n = 35$ et $p = 0,036$.
		      \item Calculer la probabilité qu'il y ait parmi les casques commandés, exactement un casque présentant un défaut de conception.
		      \item Calculer $P(X \leqslant 1)$.
	      \end{enumerate}
	\item Dans cette question, $n$ n'est pas fixé.

	      Quel doit être le nombre minimal de casques à commander pour que la probabilité
	      qu'au moins un casque présente un défaut soit supérieur à $0,99$ ?
\end{enumerate}

\lipsum
\newpage

\textbf{\textsc{Exercice 3} \quad 7 points\hfill Thèmes : suites, fonctions}

\medskip

Au début de l'année 2021, une colonie d'oiseaux comptait $40$ individus. L'observation conduit à modéliser l'évolution de la population par la suite $\left(u_n\right)$ définie pour tout entier naturel $n$ par:

\[\left\{\begin{array}{l c l}
		u_0 & = & 40 \\u_{n+}&=&0,008u_n\left(200 - u_n\right)
	\end{array}\right.\]

où $u_n$ désigne le nombre d'individus au début de l'année $(2021+n)$.

\medskip


\begin{enumerate}
	\item Donner une estimation, selon ce modèle, du nombre d'oiseaux dans la colonie au
	      début de l'année 2022.
\end{enumerate}

On considère la fonction $f$ définie sur l'intervalle [0~;~100] par $f(x) = 0,008x(200 - x)$.

\begin{enumerate}[resume]
	\item Résoudre dans l'intervalle [0~;~100] l'équation $f(x) = x$.
	\item
	      \begin{enumerate}
		      \item Démontrer que la fonction $f$ est croissante sur l'intervalle [0~;~100] et dresser son tableau de variations.
		      \item En remarquant que, pour tout entier naturel $n$,\, $u_{n+1} = f\left(u_n\right)$ démontrer par récurrence que, pour tout entier naturel $n$ :\index{récurrence}

		            \[0 \leqslant u_n \leqslant u_{n+1} \leqslant 100.\]

		      \item En déduire que la suite $\left(u_n\right)$ est convergente.\index{limite de suite}
		      \item Déterminer la limite $\ell$ de la suite $\left(u_n\right)$. Interpréter le résultat dans le contexte de l'exercice.\index{limite de suite}
	      \end{enumerate}
	\item On considère l'algorithme suivant:\index{script python}

	      \begin{center}
		      \fbox{\begin{tabular}{l}
				      def seuil(p) :                   \\
				      \qquad n=0                       \\
				      \qquad u = 40                    \\
				      \qquad while u < p :             \\
				      \quad \qquad n =n+1              \\
				      \quad \qquad u = 0.008*u*(200-u) \\
				      \qquad return(n+2021)            \\
			      \end{tabular}}
	      \end{center}

	      L'exécution de seuil(100) ne renvoie aucune valeur. Expliquer pourquoi à l'aide de la question 3.
\end{enumerate}


\end{document}
