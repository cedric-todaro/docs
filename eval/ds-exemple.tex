% vim: set ft=tex :
% arara: pdflatex
% arara: pdflatex
% arara: latexmk: { clean: partial }
% arara: clean: { files: [indent.log] }

\input{_ds.latex}

% ---------------------------------------------
\def\Classe{2nde}
\def\Titre{Évaluation n$^{\circ}$6: Calcul numérique / algébrique}
\def\NoterSur{20}

% ---------------------------------------------
\begin{document}
\DoTitle

\exo{Calcul numérique \exosur{4}}{
	\begin{enumerate}
		\q{4} Effectuer, \textbf{en détaillant}, les \np{1234} calculs suivants :
	\end{enumerate}
	$$A = 3 + 6 \times 2 - 1 \hspace{2cm} B =\cfrac{1}{9}+\cfrac{5}{6} \hspace{2cm} C = 1 - \cfrac{3}{8}\left( 1 -\frac{2}{3} \right) \hspace{2cm} D =\cfrac{\sqrt{32}}{\sqrt{2}}$$
}

\newpage

\exo{Calcul algébrique \exosur{5}}{
	\begin{enumerate}
		\q{3} Développer les expressions suivantes :
		$$A=3\left( 2x - 1 \right) \hspace{2cm} B = 5\left( 2x + 3 \right)- 2\left( x + 1 \right) \hspace{2cm} C=\left( x + 5 \right)\left( 2x - 5 \right)$$
		\q{2} Factoriser les expressions suivantes :
		$$D = 5\left( 3x + 1 \right)-5(x - 1)\hspace{2cm}E = 3\pa{x+1}+x\pa{x+1}$$
	\end{enumerate}
}

\newpage

\exo{ \exosur{3}}{
	On considère l'expression algébrique suivante : $\quad A = x^{2} - 6x + 5$
	\begin{enumerate}
		\q{1} Démontrer, en développant l'expression suivante, que : $\quad A = (x - 3)^{2} - 4$
		\q{1} Démontrer, en développant l'expression suivante, que : $\quad A = (x - 1)(x - 5)$
		\q{1} En choisissant la forme la plus adaptée, calculer la valeur de $A$ pour $x=3$, $x=1$ et $x=0$
	\end{enumerate}
}

\newpage

\exo{\exosur{3}}{
	\img{5cm}{img/image3.png}
	On considère la figure ci-contre.
	\begin{enumerate}
		\q{1} Exprimer l'aire du rectangle $A$ en fonction de $x$.
		\q{1} Exprimer l'aire du rectangle $B$ en fonction de $x$.
		\q{1} Calculer l'aire de $A$ et $B$ pour $x=0.5$
	\end{enumerate}
	\vspace{2.45cm}
}

\newpage

\exo{Fonctions du 2nd degré : reconnaissance graphique \exosur{2}}{

	\be
	\q{2} Relier chaque fonction à sa représentation graphique en justifiant brièvement votre choix.
	\ee

	\textbf{Fonctions :}

	\bi
	\item \textbf{Fonction A :\quad} $f(x) = 2x^2 - 8$
	\item \textbf{Fonction B :\quad} $g(x) = -(x + 1)(x - 3)$
	\item \textbf{Fonction C :\quad} $h(x) = -(x + 3)(x - 1)$
	\item \textbf{Fonction D :\quad} $k(x) = (x + 1)(x - 3)$
	\ei

	\vspace{0.5cm}

	\textbf{Représentations graphiques :}
	\begin{center}
		\begin{tabular}{|m{4cm}|m{4cm}|m{4cm}|m{4cm}|}\hline
			1                                      & 2 & 3 & 4 \\ \hline
			\includegraphics[width=4cm]{img/2.png} &
			\includegraphics[width=4cm]{img/4.png} &
			\includegraphics[width=4cm]{img/1.png} &
			\includegraphics[width=4cm]{img/3.png}
			\\ \hline
		\end{tabular}
	\end{center}
}

\newpage

\exo{Étude d'une fonction du second degré \exosur{5}}{

	Soit $f$ la fonction définie sur $\mathbb{R}$ par : $$f(x) = -2(x + 1)(x + 5)$$

	\be
	\q{1} Résoudre l'équation $f(x) = 0$.
	\q{1} En déduire le tableau de signes de $f(x)$.
	\q{1} Déterminer les coordonnées du sommet $S$ de la parabole représentant $f$.
	\q{1} Dresser le tableau de variations de la fonction $f$.
	\q{1} Représenter l'allure de la courbe de $f$ dans un repère orthogonal en indiquant les éléments caractéristiques (sommet, points d'intersection avec l'axe des abscisses).
	\ee
}

\newpage

\exo{Optimisation du bénéfice d'une entreprise \exosur{5}}{

	Une entreprise fabrique et vend des objets. On note $x$ le nombre d'objets produits et vendus, avec $0 \leqslant x \leqslant 130$.

	Le coût de production de $x$ objets, en euros, est modélisé par la fonction : $$C(x) = x^2 + 20x + \np{2000}$$

	Le prix de vente unitaire est de $140$\euro~donc la recette obtenue par la vente de $x$ objets, en euros, est donnée par : $$R(x) = 140x$$

	La fonction bénéfice $B(x)$ permet de modéliser le bénéfice réalisé par l'entreprise pour $x$ objets vendu et son expression est obtenu à l'aide de : $$B(x) = R(x) - C(x)$$
	\vspace*{2mm}
	\be
	\q{1} Démontrer que $B(x) = - x^2 + 120x - \np{2000}$.
	\ee

	On admet que la fonction bénéfice peut s'écrire sous la forme factorisée : $\quad B(x) = -(x - 20)(x - 100)$

	\ber
	\q{1} Démontrer par le calcul que cette forme factorisée correspond bien à l'expression de $B(x)$ trouvée à la question 1.
	\q{1} Résoudre l'équation $B(x) = 0$ et interpréter les solutions dans le contexte de l'entreprise.
	\q{1} Déterminer le tableau de variations de $B(x)$.
	\q{1} En tenant compte de la contrainte $0 \leqslant x \leqslant 130$, combien d'objets l'entreprise doit-elle produire et vendre pour réaliser le bénéfice maximum ? Quel est son montant ? Justifier votre réponse.
	\ee
}

\newpage

\exo{Probabilités \exosur{6}}{

	\textbf{Partie A : Lancer d'un dé à $4$ faces}

	On lance un dé équilibré à $4$ faces numérotées de $1$ à $4$.

	\be
	\q{0.5} Décrire l'univers $\Omega$ de cette expérience aléatoire.
	\q{0.5} Donner la loi de probabilité de cette expérience.
	\q{1} Calculer la probabilité de l'événement $A$ : « Obtenir un nombre strictement plus petit que $2$ ».
	\ee

	\vspace*{4mm}

	\textbf{Partie B : Somme de deux dés à $4$ faces}

	On lance simultanément \textbf{deux} dés équilibrés à $4$ faces numérotées de $1$ à $4$.

	On s'intéresse à la somme des résultats obtenus sur les deux dés.

	\ber
	\q{1} Compléter le tableau suivant qui permet de déterminer toutes les sommes possibles :
	\vspace*{1mm}
	\begin{center}
		\begin{tabular}{|c|c|c|c|c|} \hline
			                                       & \textbf{Dé 2 : 1} & \textbf{Dé 2 : 2} & \textbf{Dé 2 : 3} & \textbf{Dé 2 : 4} \\ \hline
			\rule[-4mm]{0mm}{9mm}\textbf{Dé 1 : 1} &                   &                   &                   &                   \\ \hline
			\rule[-4mm]{0mm}{9mm}\textbf{Dé 1 : 2} &                   &                   &                   &                   \\ \hline
			\rule[-4mm]{0mm}{9mm}\textbf{Dé 1 : 3} &                   &                   &                   &                   \\ \hline
			\rule[-4mm]{0mm}{9mm}\textbf{Dé 1 : 4} &                   &                   &                   &                   \\ \hline
		\end{tabular}
	\end{center}
	\vspace*{2mm}
	\q{1} En déduire l'univers $\Omega$ de cette expérience aléatoire.
	\q{1} Établir la loi de probabilité de la somme obtenue.
	\vspace*{1mm}
	\begin{center}
		\begin{tabular}{|c|c|c|c|c|c|c|c|} \hline
			\textbf{Somme}                            & 2            & 3            & 4            & 5            & 6            & 7            & 8            \\ \hline
			\rule[-3mm]{0mm}{8mm}\textbf{Probabilité} & \hspace{1cm} & \hspace{1cm} & \hspace{1cm} & \hspace{1cm} & \hspace{1cm} & \hspace{1cm} & \hspace{1cm} \\ \hline
		\end{tabular}
	\end{center}
	\vspace*{2mm}
	\q{1} Calculer la probabilité de l'événement $B$ : « Obtenir une somme strictement plus petite que $7$ ».
	\ee
}

\end{document}
