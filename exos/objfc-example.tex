\input{_inc.latex}

% --- Largeurs des colonnes
\def\LargColOne{10cm}
\def\LargColTwo{1.5cm}
\def\LargColThree{5.5cm}

% 
% --- Variables du doc.
% 
\def\Titre{Suites numériques}
\def\Classe{Term. Complémentaire}
\def\PageDebut{39}
\def\NbParPage{3}
% --- Le doc...
\begin{document}
\foreach \nb in {1,...,\NbParPage}{
		\ifnum\nb>1\dotfill\fi
		\begin{center}\LARGE\bfseries \Titre~(p\PageDebut+)\end{center}
		\begin{MonTableau}\hline
			Notion de limite d'une suite, opérations sur les limites de suites & 21 à 45\\ \hline
			Passage à la limite dans les inégalités. Théorème des gendarmes & 46 à 50\\ \hline
			Limite d'une suite géométrique de raison $>0$\par Limite de la somme des termes d'une suite géo. de raison $<1$ & 51 à 63\\ \hline
			Suites arithmético-géométriques : modéliser un problème par une suite donnée par une formule de récurrence & 64-65\par 67 à 71\\ \hline
			Représenter graphiquement une suite donnée par une relation de récurrence & 66-102\\ \hline
		\end{MonTableau}
		\begin{flushright}\small Nathan - Hyperbole Math Complémentaire (313-3-09-118098-0)\end{flushright}
	}
\end{document}
